\section{Introduction}
This chapter provides a review of existing research and technologies that form the foundation of the proposed Decentralized Freelance Platform. The study of related work helps identify the limitations of current systems and the areas where blockchain technology can introduce transparency, automation, and trustless collaboration. 

Several researchers and organizations have investigated the potential of blockchain to revolutionize digital marketplaces by replacing intermediaries with secure and autonomous smart contract–based systems. The following sections summarize the key contributions relevant to this project.

\section{Existing Research and Studies}
Deshmukh et al. (2020) proposed a decentralized freelancing model using the Ethereum blockchain to establish transparent and automated interactions between clients and freelancers. Their system demonstrated the use of smart contracts for secure payment handling and dispute management, eliminating the dependency on centralized servers.

Pallam and Gore (2019) developed \textit{Boomerang}, a blockchain-based freelancing paradigm built on Hyperledger. Their model emphasized scalability and privacy through permissioned ledgers, which are suitable for enterprise-grade freelance ecosystems.

The foundational contribution of blockchain technology began with Satoshi Nakamoto (2008), who introduced \textit{Bitcoin}, the first decentralized digital currency enabling peer-to-peer value exchange. Later, Wood (2013) extended this idea by introducing \textit{Ethereum}, a programmable blockchain that supports decentralized applications (DApps) through smart contracts, which became the backbone for most decentralized platforms.

The Fraunhofer ICT Group (2018) analyzed blockchain and smart contract architectures, emphasizing their ability to automate transactions securely in digital marketplaces. Similarly, Levy (2017) proposed a decentralized fundraising and freelancing network that utilized token-based incentives and community governance, paving the way for decentralized ecosystems free from intermediary control.

Chatterjee et al. (2015) explored the theoretical limits of freelance marketplaces and proposed decentralized allocation schemes to improve fairness and resource utilization in digital labor markets. This work established the conceptual groundwork for the efficient management of decentralized freelance operations.

Hald and Kinra (2019) examined blockchain’s role in improving supply chain transparency and accountability. Their observations about performance and traceability are directly applicable to freelancing platforms, which similarly depend on trust and verifiable collaboration between unknown participants.

Pourheidari et al. (2018) discussed the secure execution of untrusted business logic on permissioned blockchains, addressing issues related to smart contract execution among participants with no prior trust relationship. Their insights are valuable for ensuring security in freelance systems involving anonymous clients and workers.

Olariu et al. (2024) proposed the concept of Trust and Reputation as a Service (TRaaS), which enables reputation management in decentralized marketplaces without centralized oversight. This model directly influences the design of reputation systems for decentralized freelancing platforms.

\section{Summary}
The reviewed literature highlights that blockchain can significantly enhance the transparency, efficiency, and fairness of freelance ecosystems. By leveraging smart contracts, decentralized storage, and on-chain reputation mechanisms, blockchain enables automation of project agreements and payments while ensuring security. However, challenges such as scalability, interoperability, and user-friendly adoption remain open research issues. 

This project builds upon these prior works to design a Decentralized Freelance Platform that mitigates the limitations of centralized systems and enhances collaboration between clients and freelancers using blockchain-driven automation.
