\section{System Architecture}
The proposed system is designed using a hybrid on-chain/off-chain architecture to balance security, decentralization, and performance. The architecture consists of four primary layers:
\begin{itemize}
    \item \textbf{Blockchain Layer:} The Ethereum or Polygon blockchain serves as the foundation, hosting all smart contracts responsible for job posting, escrow management, milestone payments, and user reputation tracking. Transactions are validated through a Proof-of-Stake consensus mechanism, ensuring both security and energy efficiency.
    \item \textbf{Storage Layer:} The InterPlanetary File System (IPFS) is employed for decentralized file storage of proposals, deliverables, and related project documents. This reduces on-chain data load while maintaining data immutability and accessibility.
    \item \textbf{Application Layer:} The web interface, developed using React.js and Node.js, allows users to interact seamlessly with the blockchain. It includes functionalities such as wallet connection, job management, and milestone tracking.
    \item \textbf{Indexing Layer:} The Graph protocol is used as a middleware to index blockchain events. It allows the frontend to query blockchain data efficiently and enhances performance by avoiding direct node queries.
\end{itemize}

\section{Smart Contract Design}
Smart contracts are written in Solidity to manage the workflow between clients and freelancers. The contract governs:
\begin{itemize}
    \item \textbf{Project Creation:} A new smart contract is deployed when a client approves a freelancer’s proposal.
    \item \textbf{Escrow Logic:} Funds are locked when a project begins and released automatically after milestone approval or upon dispute resolution.
    \item \textbf{Milestone Payments:} Each project can be divided into multiple milestones, ensuring that payments are conditional and progressive.
    \item \textbf{Reputation Management:} Post-completion ratings are recorded on-chain, building an immutable reputation profile for each user.
\end{itemize}

\section{Algorithms Used}
The escrow management logic forms the core algorithm of the system:
\begin{enumerate}
    \item Client initializes a contract by locking agreed funds.
    \item Freelancer completes the assigned milestone and submits deliverables.
    \item Client verifies the work and triggers the contract’s \texttt{releasePayment()} function.
    \item Upon release, funds are transferred automatically to the freelancer’s wallet.
    \item If disputes arise, funds remain in escrow until the dispute window expires or arbitration resolves the issue.
\end{enumerate}

\section{Data Flow}
The data flow of the platform follows this sequence:
\begin{itemize}
    \item User interacts with the frontend interface to initiate a transaction.
    \item Frontend invokes blockchain smart contracts through MetaMask.
    \item Smart contracts update project state on-chain and emit event logs.
    \item The Graph middleware indexes events for quick access by the frontend.
    \item Files are uploaded and retrieved through IPFS using unique content identifiers (CIDs).
\end{itemize}

\section{System Implementation Plan}
The implementation proceeds through the following phases:
\begin{enumerate}
    \item \textbf{Phase 1:} Design and development of user interface and wallet integration.
    \item \textbf{Phase 2:} Implementation of core smart contracts for escrow and milestone handling.
    \item \textbf{Phase 3:} Integration of IPFS for decentralized file management.
    \item \textbf{Phase 4:} Deployment on Polygon test network and functional testing.
    \item \textbf{Phase 5:} Performance evaluation and comparison with centralized models.
\end{enumerate}
