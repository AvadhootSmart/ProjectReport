\section{System Architecture}

The proposed system follows a hybrid on-chain/off-chain architecture that combines the security of blockchain with the scalability of distributed storage. The architecture consists of four primary layers as shown in Figure~\ref{fig:architecture}. Each layer performs specific functions that together ensure decentralization, transparency, and robust performance.

\begin{figure}[H]
    \centering
    \includegraphics[width=0.95\textwidth]{images/architecture.png}
    \caption{Overall System Architecture of the Decentralized Freelance Platform}
    \label{fig:architecture}
\end{figure}

\noindent The major layers in the architecture are:
\begin{itemize}
    \item \textbf{Blockchain Layer:} Hosts Ethereum/Polygon smart contracts managing job posting, escrow, milestone payments, and reputation logic. Proof-of-Stake validation ensures energy efficiency and security.
    \item \textbf{Storage Layer:} IPFS (InterPlanetary File System) is used to store proposals, deliverables, and project documents in a decentralized form.
    \item \textbf{Application Layer:} A React.js and Node.js–based interface enables seamless user interactions for wallet connection, project creation, proposals, and milestone tracking.
    \item \textbf{Indexing Layer:} The Graph protocol indexes blockchain events to support optimized querying from the frontend.
\end{itemize}


\section{Use Case Diagram}

The Use Case Diagram provides a high-level view of interactions between system actors and the core functionalities of the platform. Figure~\ref{fig:usecase} shows the major actors—Client, Freelancer, and Smart Contract System—along with primary use cases such as project creation, proposal submission, milestone approval, and payment release.

\begin{figure}[H]
    \centering
    \includegraphics[width=0.85\textwidth]{images/use-case.png}
    \caption{Use Case Diagram of the Decentralized Freelance Platform}
    \label{fig:usecase}
\end{figure}

\section{Sequence Diagram}

The Sequence Diagram illustrates the chronological interaction between system components during a typical project workflow. It highlights message flow between the Client, Freelancer, Frontend Application, Smart Contract System, and IPFS. The diagram shown in Figure~\ref{fig:sequence} demonstrates the processes of project creation, proposal submission, milestone completion, payment release, and rating updates.

\begin{figure}[H]
    \centering
    \includegraphics[width=0.9\textwidth]{images/sequence.png}
    \caption{Sequence Diagram of the Decentralized Freelance Platform}
    \label{fig:sequence}
\end{figure}


\section{Activity Diagram}

The Activity Diagram represents the end-to-end workflow of the platform from the perspective of user actions and system responses. It captures major activities such as job posting, proposal submission, milestone approval, payment settlement, and final review generation. Figure~\ref{fig:activity} describes the complete operational flow including decision points, parallel actions, and automated contract triggers.

\begin{figure}[H]
    \centering
    \includegraphics[width=0.2\textwidth]{images/activity.png}
    \caption{Activity Diagram of the Workflow in the Decentralized Freelance Platform}
    \label{fig:activity}
\end{figure}


\section{Data Flow Diagrams (DFD)}

The Data Flow Diagrams illustrate how data moves through the system during different operations.

\subsection{DFD Level 0}

Figure~\ref{fig:dfd0} illustrates the Level 0 DFD, presenting the system as a single entity interacting with two external actors—Client and Freelancer. It captures the high-level flow of data entering and exiting the system.

\begin{figure}[H]
    \centering
    \includegraphics[width=0.9\textwidth]{images/dfd0.png}
    \caption{Level 0 Data Flow Diagram of the System}
    \label{fig:dfd0}
\end{figure}

\subsection{DFD Level 1}

Figure~\ref{fig:dfd1} shows the Level 1 DFD, describing internal processes including job posting, proposal submission, escrow initiation, milestone processing, and reputation updates. It highlights how the application layer interacts with blockchain smart contracts and IPFS.

\begin{figure}[H]
    \centering
    \includegraphics[width=0.9\textwidth]{images/dfd1.png}
    \caption{Level 1 Data Flow Diagram showing subsystem interactions}
    \label{fig:dfd1}
\end{figure}


\section{Smart Contract Design}

Smart contracts written in Solidity form the core automation component of the system. They ensure transparent and tamper-proof enforcement of project agreements. The major functions include:

\begin{itemize}
    \item \textbf{Project Creation:} Upon acceptance of a proposal, a new smart contract instance is deployed.
    \item \textbf{Escrow Logic:} Client funds are locked in the contract and released only after verification of completed milestones.
    \item \textbf{Milestone Payments:} Each milestone is verified individually, and payments are executed through the \texttt{releasePayment()} function.
    \item \textbf{Reputation Management:} Ratings and reviews are stored on-chain to maintain immutable user credibility.
\end{itemize}


\section{Algorithms Used}

The escrow automation algorithm ensures fairness and secure fund handling throughout the project lifecycle:

\begin{enumerate}
    \item The client initializes the contract and deposits the agreed funds.
    \item The freelancer completes a milestone and submits deliverables.
    \item The client verifies and triggers \texttt{releasePayment()}.
    \item Upon approval, funds are transferred automatically to the freelancer.
    \item In case of disputes, funds remain locked until arbitration or timeout resolution.
\end{enumerate}


\section{Data Flow within the System}

The operational data flow can be summarized as follows:
\begin{itemize}
    \item User interacts with the React.js frontend to initiate actions.
    \item The frontend uses MetaMask to sign and send blockchain transactions.
    \item Smart contracts update blockchain state and emit events.
    \item The Graph indexes these events for efficient retrieval.
    \item IPFS stores and retrieves deliverable files using unique CIDs (Content Identifiers).
\end{itemize}


\section{System Implementation Plan}

The development lifecycle of the system is divided into sequential phases:

\begin{enumerate}
    \item \textbf{Phase 1:} Build frontend UI and integrate MetaMask wallet.
    \item \textbf{Phase 2:} Develop escrow and milestone smart contracts.
    \item \textbf{Phase 3:} Integrate IPFS for decentralized storage.
    \item \textbf{Phase 4:} Deploy and test on Polygon Mumbai Testnet.
    \item \textbf{Phase 5:} Evaluate performance and compare with centralized alternatives.
\end{enumerate}
